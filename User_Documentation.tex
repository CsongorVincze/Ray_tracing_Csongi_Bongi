\documentclass{article}

% --- PACKAGES ---
\usepackage[utf8]{inputenc} % Allows standard characters
\usepackage{geometry}       % Adjusts page margins
\usepackage{listings}       % ESSENTIAL: Allows you to format code blocks
\usepackage{xcolor}         % Adds color support for the code blocks
\usepackage{hyperref}       % Makes the Table of Contents clickable

% --- CODE BLOCK STYLING ---
% This sets up how your code will look (like a dark theme editor)
\lstset{
    backgroundcolor=\color{black!5},   % Light grey background
    basicstyle=\ttfamily\small,        % Typewriter font
    breaklines=true,                   % Break long lines automatically
    frame=single,                      % Adds a frame around the code
    keywordstyle=\color{blue},         % Keywords (if/else) in blue
    commentstyle=\color{green!50!black}% Comments in dark green
}

% --- TITLE PAGE DATA ---
\title{Sugárkövető felhasználói dokumentáció}
\author{Vincze Csongor}
\date{\today}

% --- DOCUMENT BEGINS HERE ---
\begin{document}

% 1. Create Title Page
\maketitle


% 3. Main Content
\section{A program rövid leírása}
Ez a program egy nagyon alapvető sugárkövetőt (ray tracer) implementál le.
A lényege, hogy a virtuális kamerából (ahonnét mi a képet látjuk) sugarakat küld,
és ezeknek a sugaraknak a gömbökkel való ütközöését követi.
Miután a felhasználó megadja a megfelelő paramétereket a program elkezdi a képkockák
generálását. A képkockák generálásának előrehaladását közben a felhasználó egy 
folyamatjelző sávon követheti. Amint az összes képkocka generálása megtörtént
a szoftver összefűzi a képkockákat egy videóvá, és ezt el is indítja.


\section{Installáció}
A program (és annak annak minden része) erről a linkről érhető el:

\url{https://github.com/CsongorVincze/Ray_tracing_Csongi_Bongi}

A \texttt{video\_render.exe} készen futtatható. (A többi fájl nagyrésze a forráskódhoz tartozik.)
A program csak windowson működik jelenleg.

\section{A program kezelése}
A \texttt{video\_render.exe} futtatásakor a program bekéri a videó
elkészítéséhez szükéges adatokat. Ezek sorba:
\begin{enumerate}
    \item Milyen felbontást szeretnél? (szélesség pixelekben) (elfogadható értékek: $2^n$, ahol $n$ egész és $4 < n < 11$)
    \item Hány gömbot szeretnél? (ajánlott: 2-10)
    \item Milyen hosszú videót szeretnél? (a videó 30fps-en fut, 4-14 közötti egész szám elfogadott)
\end{enumerate}

Az első bemenet a kép vízszíntes pixelszámát adja meg (a képarány 16:9). Itt 1024-nél már jelentősen
lelassul a program. 256-nál viszont még elég alacsony lesz a készített videó minősége.
Érdemes lehet ezt az értéket 512-re állítani.
A gömbök száma egészen 0-tól 50-ig terjedhet. Ennek ellenére nem ajánlott 10-nél
többet megadni mivel ez is jelentősen lelassítja a programot. A harmadik bekérésnél hasonló
okokból kiindulva 5 körüli értéket érdemes megadni.

Jó szórakozást!

\end{document}