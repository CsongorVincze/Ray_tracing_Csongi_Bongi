\documentclass{article}

% --- PACKAGES ---
\usepackage[utf8]{inputenc} % Allows standard characters
\usepackage{geometry}       % Adjusts page margins
\usepackage{listings}       % ESSENTIAL: Allows you to format code blocks
\usepackage{xcolor}         % Adds color support for the code blocks
\usepackage{hyperref}       % Makes the Table of Contents clickable

% --- CODE BLOCK STYLING ---
% This sets up how your code will look (like a dark theme editor)
\lstset{
    backgroundcolor=\color{black!5},   % Light grey background
    basicstyle=\ttfamily\small,        % Typewriter font
    breaklines=true,                   % Break long lines automatically
    frame=single,                      % Adds a frame around the code
    keywordstyle=\color{blue},         % Keywords (if/else) in blue
    commentstyle=\color{green!50!black}% Comments in dark green
}

% --- TITLE PAGE DATA ---
\title{Fejlesztői dokumentáció}
\author{Vincze Csongor}
\date{\today}

% --- DOCUMENT BEGINS HERE ---
\begin{document}

% 1. Create Title Page
\maketitle
\newpage

% 2. Create Table of Contents
\tableofcontents
\newpage

% 3. Main Content
\section{A program ismertetése, áttekintése}
Ez a program egy nagyon alapvető sugárkövetőt (ray tracer) implementál le.
A lényege, a virtuális kamerából (ahonnét mi a képet látjuk) sugarakat küld,
és ezeknek a sugaraknak a gömbökkel való ütközöését követi.
Miután a felhasználó megadja a megfelelő paramétereket a program elkezdi a képkockák
generálását. A képkockák generálásának előrehaladását közben a felhasználó egy 
folyamatjelző sávon követheti. Amint az összes képkocka generálása megtörtént
a szoftver összefűzi a képkockákat egy videóvá, és ezt el is indítja.

\section{Nyelv, használt csomagok, kiegészítők}
A teljes projekt C nyelvben íródott, az alapvető C könyvtárak felhasználásával,
minden külső csomag nélkül. A program nagy vonalakban a \url{https://raytracing.github.io/books/RayTracingInOneWeekend.html#overview}
dokumentációt követi (természetesen C nyelvre átírva).

\section{Az arhitektura áttekintése, a program kezelése}
A \texttt{video\_render.exe} futtatásakor a program bekéri a videó
elkészítéséhez szükéges adatokat. Ezek sorba:
\begin{enumerate}
    \item Milyen felbontást szeretnél? (szélesség pixelekben) (elfogadható értékek: $2^n$, ahol $n$ egész és $4 < n < 11$)
    \item Hány gömbot szeretnél? (ajánlott: 2-10)
    \item Milyen hosszú videót szeretnél? (a videó 30fps-en fut, 1-10 közötti egész szám elfogadott)
\end{enumerate}

Az első bemenet a kép vízszíntes pixelszámát adja meg (a képarány 16:9). Itt 1024-nél már jelentősen
lelassul a program. 256-nál viszont még elég alacsony lesz a készített videó minősége.
Érdemes lehet ezt az értéket 512-re állítani. Ezzel az adattal a program
már el tudja készíteni az ún. nézőportált (viewport), ami meghatározza, hogy mit, hogy fogunk látni. A többi paraméter
fix. A \texttt{viewport\_creator} függvény végzi ezt a folyamatot, és a \texttt{video\_render.c} hívja meg a \texttt{render.h}-ból.
A gömbök száma egészen 0-tól 50-ig terjedhet. Ennek ellenére nem ajánlott 10-nél
többet megadni mivel ez is jelentősen lelassítja a programot. A gömbökre van egy külön struktúra (sphere), egy ilyen tömböt hozunk
létre. A a videó készítés közben mozgatni fogjuk a gömböket. Mivel csak gömb alakú objektumokkat dolgozunk,
így érdemes ennek a tömbnek az elejére egy előre beállított nagy gömböt rakni, hogy
legyen egy földünk a videókhoz. Így valójában egyel több gömbbel operálunk mintamennyit bekértünk.
A harmadik bekérésnél hasonló
okokból kiindulva 5 körüli értéket érdemes megadni. Mivel konstans 30 fps-el dolgozunk így csak simán visszaszámoljuk,
hogy hány képkockát kell legeneráljunk.


\section{Fájlok és azok leírásaik}
A program egy fő .c fájlból és 10 header fájlból áll. A header fájlok főleg a program egyes részeihez
biztosítanak függvényeket. Minden fájl ugyan abba n a mappában van.

    \subsection{vec\_3.h}
    Mivel 3 dimenzióban dolgozunk szükségünk van vektoroperációkra.
    Ez a fájl biztosítja ezek meglétét. A fájlban minden függvény mellé fel van tüntetve, hogy mit csinál,
    így ezt nem részletezem itt. Ez a fájl tartalmazza a \texttt{vec\_3.h} struktúrát is ami, lényegében egy 3 dimenziós vektor típus.

    \subsection{ray.h}
    Ez a rész csupán a sugár (ray) struktúrát és egz sugárkövetőt tartalmazza.
    A sugár egy kiinduló pontból és egy irányvektorból áll. a sugárkövető pedig visszaadja
    egy t paraméter függvényében, hogy hol van a sugár.

    \subsection{sphere.h}
    Ez a fájl a gömb struktúrát hozza létre. A gömb egy középpontból és egz sugárból áll.
    Ezután a \texttt{hit\_sphere} függvény következik ami, lényegileg eldönti, hogy az adott sugár talált-e és ha igen melyik
    gömböt találta el.
    %itt ezt ki kell felyteni es csekkolni
    Ezt két függvény követi; az \texttt{\_array\_spheres} és a \texttt{\_rand\_spheres}. Előbbi egy vízszintes sorba rak le gömböket.
    Útóbbi pedig (bizonyos korlátok kötött) random paraméterekkel rakja le a gömböket.






\end{document}